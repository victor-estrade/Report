%% Template pour rapport de stage du master AIC (Apprentissage,
%% Information et Contenu), Université Paris-Saclay
%% Template d'origine : 
%% Copyright (C) 2008 Johan Oudinet <oudinet@lri.fr>
%%  
%% Permission is granted to make and distribute verbatim copies of
%% this manual provided the copyright notice and this permission notice
%% are preserved on all copies.
%%  
%% Permission is granted to process this file through TeX and print the
%% results, provided the printed document carries copying permission
%% notice identical to this one except for the removal of this paragraph
%% (this paragraph not being relevant to the printed manual).
%%  
%% Permission is granted to copy and distribute modified versions of this
%% manual under the conditions for verbatim copying, provided that the
%% entire resulting derived work is distributed under the terms of a 
%% permission notice identical to this one.
%%  
%% Permission is granted to copy and distribute translations of this manual
%% into another language, under the above conditions for modified versions,
%% except that this permission notice may be stated in a translation
%% approved by the Free Software Foundation
%%  

% Contexte, objectifs, contraintes

\chapter{Contexte}



Cependant on ne dispose pas nécessairement de la correspondance entre un élément $x_{src}$ de 
la \source{} et ce qu'il aurait été dans l'espace \cible{} $x_{tgt}$.

On a 3 cas dans l'ordre croissant de difficulté:
\begin{itemize}
\item On dispose de la correspondance $x_{src}\to x_{tgt}$. 
	Il s'agit d'un cas d'apprentissage supervisé (régression multiple, facile ?).
\item On a un paquet de donnée \source{} et leur correspondant exacte dans la \cible{}.
	Cependant on ignore quel élément $x_{src}$ correspond à quel autre élément $x_{tgt}'$.
\item On a un jeu de donnée \source{} et un jeu de donnée \cible{} 
	qui ne sont pas les homologues des données \sources{}.
\end{itemize}
+ des mélanges de ces 3 cas.


\section{Prop} % (fold)
\label{sec:prop}

L'objectif est un plongement $g$ de l'espace \source{} sur l'espace \cible{}; tel que:
\begin{itemize}
\item Si l'on dispose d'un appariement un à un ($x_{source}$ et $f(x_{source}) = x_{target}$),
	le plongement réalise $g = f^{-1}$
\item Si l'on dispose de classes définies sur source et cible,
	ces classes sont préservées par le plongement, 
	i.e. $\text{classe}(x_{source}) = \text{classe}(g(x_{source}))$
\item Distribution $g(source)$ = distribution cible 
	(ou du moins, ces deux distributions ne sont pas distinguables).
\end{itemize}

Nous imposons le plongement en imposant que la partie cible sur l'intermédiaire soit l'identité.

% section prop (end)


\TODO Présenter les exemples de problème qu'on va traiter et ceux qu'on aurait aimé traiter.



\section{Comparaison avec DANN} % (fold)
\label{sec:comparaison_avec_dann}

\TODO Dire en quoi ça ressemble au DANN et en quoi c'est différent.
\TODO (De toute manière on fait les expériences avec et sans DANN pour voir)

Ce que fait DANN:\\
chercher une représentation commune de la \source{} et de la \cible{} ne détruisant pas 
l'information nécessaire à la tâche demandée au réseau de neurone (classification ou autre).

Ce qu'il ne fait pas:\\
La représentation commune donne quoi ? \TODO


% section comparaison_avec_dann (end)



%%% Local Variables: 
%%% mode: latex
%%% TeX-master: "rapportM2R"
%%% End: 
