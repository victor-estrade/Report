%% Template pour rapport de stage du master AIC (Apprentissage,
%% Information et Contenu), Université Paris-Saclay
%% Template d'origine : 
%% Copyright (C) 2008 Johan Oudinet <oudinet@lri.fr>
%%  
%% Permission is granted to make and distribute verbatim copies of
%% this manual provided the copyright notice and this permission notice
%% are preserved on all copies.
%%  
%% Permission is granted to process this file through TeX and print the
%% results, provided the printed document carries copying permission
%% notice identical to this one except for the removal of this paragraph
%% (this paragraph not being relevant to the printed manual).
%%  
%% Permission is granted to copy and distribute modified versions of this
%% manual under the conditions for verbatim copying, provided that the
%% entire resulting derived work is distributed under the terms of a 
%% permission notice identical to this one.
%%  
%% Permission is granted to copy and distribute translations of this manual
%% into another language, under the above conditions for modified versions,
%% except that this permission notice may be stated in a translation
%% approved by the Free Software Foundation
%%  

% Contexte, objectifs, contraintes

\chapter{Contexte}


\chapter{Objectif, motivations}
baratin
objectif est un plongement $g$ de l'espace source sur l'espace cible; tel que:
liste des propriétés attendues:
\begin{itemize}
\item Si l'on dispose d'un appariement un à un ($x_source$ et $f(x_source)$), le plongement réalise $g = f^{-1}$
\item Si l'on dispose de classes définies sur source et cible, ces classes sont préserv'ees par le plongement, i.e. classe $(x_source)$ = classe$(g(x_source))$
\item Distribution g(source) = distribution cible (ou du moins, ces deux distributions ne sont pas distinguables).
\end{itemize}

Nous imposons le plongement en imposant que la partie cible sur l'intermediaire soit l'identité 

%%% Local Variables: 
%%% mode: latex
%%% TeX-master: "rapportM2R"
%%% End: 
