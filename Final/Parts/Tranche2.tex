\chapter{Adaptation de domaine et transport optimal}
\label{chap:tranche2}

La solution présentée dans le chapitre précédent présente plusieurs limitations, liées à son passage à l'échelle par rapport au nombre de clusters nécessaires pour représenter les données en grande dimension. Par ailleurs, l'appariement heuristique des clusters réalisés s'apparente au transport optimal. Rappelons toutefois (chapitre \ref{chap:SOA}) que le transport optimal dispose d'une matrice de coût définie par l'expert, dont nous ne disposons pas. 

Il vient ainsi naturellement de proposer l'algorithme hybride \XX, effectuant de manière itérative inspirée de l'algorithme EM \cite{EM} l'estimation de la matrice de coût, et l'optimisation du plongement admissible répondant à cette matrice de coût.

\section{Contexte 4: plongement admissible avec estimation de transport optimal}

Cette partie décrit le pseudo-code de l'algorithme.

\TODO Chaque étape.

\section{Résultats}


