
\labday{Mardi, 12 juin 2016}
\label{day:12-06-2016}

\experiment{Compréhension du problème de Transport Optimal}


On dispose de deux distributions de probabilité $\mathcal{P}_S$ et $\mathcal{P}_T$.
On désire projeter la distribution $\mathcal{P}_S$ sur $\mathcal{P}_T$.
Ces distributions sont équipées d'un histogramme (vecteur qui somme à 1) $h^S$ et $h^T$.
Dans la pratique on dispose de données tirées de ces distributions et pas les 
distributions elles même. 
On a $\mathcal{X}_S$ contenant $n_S$ points tirées de $\mathcal{P}_S$.
Ainsi que $\mathcal{X}_T$ contenant $n_T$ points tirées de $\mathcal{P}_T$.

Exemple : $\forall i, h^S_i = (1/n_S)$ et $\forall j, h^T_j = (1/n_T)$

ou encore : $\forall i, h^S_i = \sum_k e^{-\gamma ||x_i^S-x_k^S||^2}$ et $\forall j, h^T_j = \sum_k e^{-\gamma ||x_j^T-x_k^T||^2}$

Le problème de transport optimal consiste à envoyer la distribution $\mathcal{P}_S$ 
sur $\mathcal{P}_T$ à moindre coût.

On a besoin d'une matrice de coût $C \in \mathbb{R}^{n_S\times n_T}$ dont 
l'élément $c_{ij}$ indique le coût d'envoyer $x_i^S$ sur $x_j^T$ 
(= le coût d'envoyer $h^S_i$ sur $h^T_j$)

On cherche à résoudre : 
\begin{equation}
\min_T \left<T,C\right> = \min_{T} \sum_{ij} t_{ij}*c_{ij}
\end{equation}
avec
\begin{equation}
\forall j, \sum_i t_{ij} = h^T_j 
\label{mass_in}
\end{equation}
\begin{equation}
\forall i, \sum_j t_{ij} = h^S_i
\label{mass_out}
\end{equation}
\begin{equation}
\sum_i h^S_i = 1
\label{mass_in_tot}
\end{equation}
\begin{equation}
\sum_j h^T_j = 1	
\label{mass_out_tot}
\end{equation}

l'équation \eqref{mass_in} signifie que la somme de la masse des éléments de $\mathcal{X}_S$
envoyés sur $x^T_j$ vaut $h^T_j$, ie $x^T_j$ ne reçoit pas plus que sa capacité.

l'équation \eqref{mass_out} signifie que la somme de la masse de $x^S_i$ envoyé dans les 
éléments de $\mathcal{X}_T$ vaut $h^S_i$, ie $x^S_i$ ne donne pas plus qu'il ne possède.



\subexperiment{Cas 2 points 2D}

On traite le cas simple de 2 points $A\ (x_A, y_A)$ et $B\ (x_B, y_B)$ 
que l'on veux projeter sur $A^\prime\ (x_{A^\prime}, y_{A^\prime})$ et $B\ (x_{B^\prime}, y_{B^\prime})$ 

On a une matrice de coût :
$$
C = 
\begin{pmatrix}
   c_{AA^\prime} & c_{AB^\prime} \\
   c_{BA^\prime} & c_{BB^\prime} 
\end{pmatrix}
$$

on cherche : 
\begin{equation}
\min_T \left<T,C\right> = \min_{T} 
t_{AA^\prime}*c_{AA^\prime} + 
t_{AB^\prime}*c_{AB^\prime} + 
t_{BA^\prime}*c_{BA^\prime} + 
t_{BB^\prime}*c_{BB^\prime}
\end{equation}

\begin{equation}
\left \{
\begin{array}{r c l}
  t_{AA^\prime} + t_{AB^\prime}  & = & 1 \\
  t_{BA^\prime} + t_{BB^\prime}  & = & 1 \\
  t_{AA^\prime} + t_{BA^\prime}  & = & 1 \\
  t_{AB^\prime} + t_{BB^\prime}  & = & 1 \\
\end{array}
\right .
\text{d'où}
\left \{
\begin{array}{r c l}
  t_{AA^\prime}  & = & 1 - t_{AB^\prime}\\
  t_{BB^\prime}  & = & 1 - t_{BA^\prime}\\
  t_{AA^\prime}  & = & 1 - t_{BA^\prime}\\
  t_{AB^\prime}  & = & 1 - t_{BB^\prime}\\
\end{array}
\right .
\text{donc}
\left \{
\begin{array}{r c l}
  t_{AA^\prime}  & = & t_{BB^\prime}\\
  t_{AB^\prime}  & = & t_{BA^\prime}\\
  t_{AA^\prime}  & = & 1 - t_{BA^\prime}\\
\end{array}
\right .
\end{equation}

on définit : $\rho = t_{AA^\prime} = t_{BB^\prime}$ on a $1-\rho = t_{AB^\prime} = t_{BA^\prime}$ et $0\leq \rho\leq 1$

ce qui donne :
$$
T = 
\begin{pmatrix}
   \rho & 1-\rho \\
   1- \rho & \rho
\end{pmatrix}
$$
et le problème devient
\begin{equation}
\min_\rho \rho c_{AA^\prime} + (1- \rho) c_{AB^\prime} + (1-\rho) c_{BA^\prime} + \rho c_{BB^\prime}
\end{equation}
\begin{equation}
\min_\rho \rho (c_{AA^\prime} + c_{BB^\prime} - c_{AB^\prime} - c_{BA^\prime}) + 2
\end{equation}

au final :

\begin{equation}
\rho = \left \{
\begin{array}{r c l}
	1\ \text{si}\ c_{AA^\prime} + c_{BB^\prime} \leq c_{AB^\prime} + c_{BA^\prime} \\
	0\ \text{si}\ c_{AA^\prime} + c_{BB^\prime} \geq c_{AB^\prime} + c_{BA^\prime} \\
\end{array}
\right .
\end{equation}




%----------------------------------------------------------------------------------------
